% ------------------------- LITERATURE REVIEW --------------------------------
\chapter{LITERATURE REVIEW}%
\label{chap:literature_review}

“SMS spam detection has evolved from rule-based systems to data-driven and explainable machine learning approaches.”

% Overview of SMS Spam Detection
\section{OVERVIEW OF SMS SPAM DETECTION}

SMS spam detection has been an active research area due to the increasing misuse of mobile 
communication channels for unsolicited and fraudulent activities. Early approaches relied heavily 
on rule-based systems and manually crafted keyword filters. Although effective for simple spam 
patterns, such systems lacked adaptability and failed to generalize against evolving spam 
strategies~\cite{almeida2011contributions}.

With the growth of labeled SMS datasets, machine learning-based text classification techniques 
became the dominant approach. These systems leverage statistical properties of text to learn 
discriminative patterns between legitimate and malicious messages.

% Traditional Machine Learning Approaches
\section{TRADITIONAL MACHINE LEARNING APPROACHES}

% Naive Bayes Classifier
\subsection{Naive Bayes Classifier}

The Naive Bayes (NB\nomenclature{NB}{Naive Bayes}) classifier is one of the most widely used 
algorithms for text classification due to its simplicity and computational efficiency. It assumes 
conditional independence between words given the class label. Despite this strong assumption, 
Naive Bayes has shown competitive performance in SMS spam filtering tasks, particularly when 
combined with bag-of-words or TF-IDF~\nomenclature{TF-IDF}{Term Frequency-Inverse Document Frequency} 
representations~\cite{mcCallum1998comparison}.

However, NB models are limited in capturing contextual relationships between words, which restricts 
their effectiveness in detecting sophisticated spam messages such as phishing attempts.

% Logistic Regression
\subsection{Logistic Regression}

Logistic Regression (LR\nomenclature{LR}{Logistic Regression}) is a discriminative linear model 
commonly applied in binary and multi-class text classification. It estimates class probabilities 
directly and is less sensitive to irrelevant features when regularization is applied. LR-based 
spam filters have demonstrated stable and interpretable performance in SMS classification 
tasks~\cite{almeida2011contributions}.

The linear nature of Logistic Regression limits its ability to model non-linear patterns inherent 
in complex spam messages.

% Support Vector Machines
\subsection{Support Vector Machines}

Support Vector Machines (SVMs) are margin-based classifiers 
that aim to maximize the separation between classes. SVMs have been extensively used for spam 
detection due to their robustness in high-dimensional feature spaces~\cite{cortes1995support}. 
When combined with TF-IDF features, SVMs often outperform simpler probabilistic models.

Despite their effectiveness, SVMs suffer from high computational cost during training and lack 
inherent probabilistic interpretability, which poses challenges for explainability.

% Deep Learning Approaches for SMS Classification
\section{DEEP LEARNING APPROACHES FOR SMS CLASSIFICATION}

% Recurrent Neural Networks
\subsection{Recurrent Neural Networks}

Recurrent Neural Networks (RNNs) are designed to model sequential data by maintaining temporal 
dependencies across inputs. In the context of SMS classification, RNNs capture word order and 
contextual information that traditional bag-of-words models ignore.

Long Short-Term Memory (LSTM) networks address the vanishing gradient problem in standard RNNs 
and have demonstrated improved performance in text classification tasks~\cite{hochreiter1997long}. 
However, deep learning models require larger datasets and are often criticized for their 
black-box behavior.

% Ensemble Learning Techniques
\section{ENSEMBLE LEARNING TECHNIQUES}

Ensemble learning combines multiple models to improve generalization and robustness. Techniques 
such as voting, averaging, and stacking leverage the strengths of individual classifiers while 
mitigating their weaknesses~\cite{dietterich2000ensemble}. In spam detection, ensemble models have 
been shown to outperform single-model approaches, particularly when datasets contain diverse 
message patterns.

This project adopts an ensemble-inspired strategy by aggregating predictions from classical and 
deep learning models.

% Explainable Artificial Intelligence (XAI)
\section{EXPLAINABLE ARTIFICIAL INTELLIGENCE (XAI)}

% Need for Explainability
\subsection{Need for Explainability}

As machine learning systems increasingly influence user-facing decisions, explainability has become a 
critical requirement. Black-box models undermine user trust and complicate debugging, auditing, 
and regulatory compliance~\cite{sommerville2011software}.

% Model-Agnostic Explanation Techniques
\subsection{Model-Agnostic Explanation Techniques}

Local Interpretable Model-agnostic Explanations (LIME) generate local surrogate models to explain 
individual predictions by approximating the decision boundary around a specific 
instance~\cite{ribeiro2016why}. Similarly, SHAP values provide a unified framework for feature 
attribution based on cooperative game theory~\cite{lundberg2017unified}.

These techniques are particularly suitable for SMS classification, as they allow token-level 
interpretation regardless of the underlying model.

% Research Gap and Justification
\section{RESEARCH GAP AND JUSTIFICATION}

From the reviewed literature, the following gaps are identified:
\begin{itemize}
    \item Most existing works evaluate accuracy but do not assess interpretability quality.
    \item Limited focus on multi-category SMS spam classification.
    \item Lack of integrated explainability in ensemble-based SMS filters.
    \item Insufficient emphasis on user-understandable explanations in deployed systems.
\end{itemize}
\enumitemEndSpacing{}

The proposed \textbf{\Project{}} addresses these gaps by combining multi-model classification with 
explainable AI techniques in a unified framework.
