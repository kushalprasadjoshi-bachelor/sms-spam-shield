% ------------------------- REQUIREMENT ANALYSIS ------------------------------
\chapter{REQUIREMENT ANALYSIS}%
\label{chap:requirement_analysis}

This chapter presents the requirement analysis for the proposed \Project{} system.
Requirement analysis is a critical phase of the software development lifecycle that focuses on
identifying, analyzing, and documenting the functional and non-functional requirements of the
system. A clear understanding of system requirements ensures that the developed solution meets
user expectations, remains feasible, and achieves its intended objectives.

\section{FEASIBILITY STUDY}

A feasibility study was conducted to evaluate the practicality and viability of the proposed
system before development. The study considers technical, economic, operational, and time
feasibility aspects.

\subsection{Technical Feasibility}
The proposed system is technically feasible as it relies on well-established machine learning
and deep learning techniques. The required tools, libraries, and frameworks such as Python,
scikit-learn, TensorFlow, and explainability libraries are freely available and widely supported.
The computational requirements are moderate and can be fulfilled using standard personal
computing hardware.

\subsection{Economic Feasibility}
The system is economically feasible since it uses open-source software tools and publicly
available datasets. No proprietary software or paid services are required, minimizing overall
development costs.

\subsection{Operational Feasibility}
The system is designed to be user-friendly and does not require extensive technical expertise
to operate. The classification results and explanations are presented in an understandable
format, ensuring acceptance by end users and stakeholders.

\subsection{Time Feasibility}
The project is feasible within the given academic timeline. The use of an incremental
development model allows the system to be developed in stages, ensuring timely completion
of each module.

\section{FUNCTIONAL REQUIREMENTS}

Functional requirements describe the specific functionalities and services that the proposed
system must provide. The key functional requirements of the system are as follows:
\begin{itemize}
    \item The system shall allow users to submit SMS messages and view predictions in real time using web interface.
    \item The system shall preprocess input messages using tokenization, stop-word removal,
          and lemmatization techniques.
    \item The system shall extract textual features using Bag-of-Words and TF-IDF methods.
    \item The system shall classify SMS messages into multiple categories such as legitimate,
          promotional spam, and phishing messages.
    \item The system shall implement classical machine learning classifiers including Naive
          Bayes, Logistic Regression, and Support Vector Machine.
    \item The system shall support deep learning models such as Recurrent Neural Networks
          or Long Short-Term Memory networks.
    \item The system shall aggregate predictions from multiple models using an ensemble
          strategy.
    \item The system shall generate explainable outputs using model-agnostic explanation
          techniques such as LIME and SHAP.\
    \item The system shall display classification results along with token-level explanations.
    \item The system shall allow inference on previously unseen SMS messages.
\end{itemize}
\enumitemEndSpacing{}

\section{NON-FUNCTIONAL REQUIREMENTS}

Non-functional requirements define the quality attributes and performance constraints of the
system. The following non-functional requirements are identified:

\subsection{Performance Requirements}
The system shall classify SMS messages with acceptable response time suitable for real-time
or near real-time usage.

\subsection{Accuracy and Reliability}
The system shall provide consistent and reliable classification results across different SMS
categories with minimal classification errors.

\subsection{Usability Requirements}
The system interface shall be simple and intuitive, allowing users to easily input messages
and understand the output and explanations without prior training.

\subsection{Scalability Requirements}
The system shall be capable of handling an increasing number of SMS messages without
significant degradation in performance.

\subsection{Security Requirements}
The system shall ensure that input SMS data is processed securely and is not stored or shared
without authorization.

\subsection{Maintainability Requirements}
The system shall be modular in design, allowing easy updates, model replacement, and
extension of functionality in future iterations.

\subsection{Portability Requirements}
The system shall be portable and capable of running on different operating systems such as
Windows and Linux with minimal configuration changes.

\section{Hardware Requirements}

The minimum hardware requirements for implementing the proposed system are:
\begin{itemize}
    \item Processor: Intel Core i5 or equivalent
    \item RAM\texttt{:} Minimum 8 GB
    \item Storage: Minimum 10 GB free disk space
    \item Optional GPU support for deep learning model training
\end{itemize}
\enumitemEndSpacing{}

\section{Software Requirements}

The software requirements for the proposed system are as follows:
\begin{itemize}
    \item Operating System: Windows / Linux / MacOS
    \item Programming Language: Python (version 3.x)
    \item Development Environment: Jupyter Notebook or any Python IDE (VS code)
    \item Frontend Technologies: HTML, CSS, and JavaScript
    \item Backend Framework: Flask or equivalent Python web framework
\end{itemize}
\enumitemEndSpacing{}

\section{Tools, Libraries, and Environment}

The following tools and libraries are used in the development of the system:
\begin{itemize}
    \item HTML and CSS for designing the web-based user interface
    \item JavaScript for client-side interactivity
    \item Flask for integrating the frontend with the backend
    \item NumPy and Pandas for data manipulation
    \item Scikit-learn for classical machine learning models
    \item TensorFlow (KerasAPI) for deep learning models
    \item NLTK for text preprocessing
    \item LIME for local interpretability
    \item SHAP for global and local model explanations
    \item Matplotlib and Seaborn for visualization
\end{itemize}
\enumitemEndSpacing{}

\section{Summary}

This chapter presented the requirement analysis of the proposed \Project{} system. It discussed the feasibility of the project, identified functional and non-functional
requirements, and specified the necessary hardware, software, tools, and libraries. A clear
definition of these requirements provides a strong foundation for the subsequent design,
implementation, and evaluation phases of the project.
% -----------------------------------------------------------------------------------------------
