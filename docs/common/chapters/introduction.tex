% --------------------- INTRODUCTION -------------------------------
\chapter{INTRODUCTION}%
\label{chap:introduction}

% Background and Motivation
\section{BACKGROUND AND MOTIVATION}

Short Message Service (SMS) remains one of the most widely used communication mediums due to its 
simplicity, low cost, and universal availability across mobile devices. However, this widespread 
adoption has also made SMS an attractive channel for unsolicited and malicious messages, including 
spam, phishing attempts, and fraudulent promotions. Traditional SMS filtering solutions often 
focus on binary classification: labeling messages as either spam or legitimate; which is 
increasingly insufficient in modern threat landscapes~\cite{almeida2011contributions}.

Recent advances in machine learning have enabled more accurate text classification techniques, 
yet most deployed systems operate as black boxes, offering limited insight into why a particular 
message was flagged. This lack of transparency raises concerns regarding trust, accountability, 
and regulatory compliance, especially when automated systems influence user 
communication~\cite{sommerville2011software}. These challenges motivate the development of an 
intelligent, transparent, and multi-category SMS spam detection system.

% Problem Statement
\section{PROBLEM STATEMENT}

Existing SMS spam detection systems suffer from three primary limitations:
\begin{enumerate}
    \item \textbf{Binary classification constraint:} Most systems classify SMS messages only as 
        spam or non-spam, failing to distinguish between different spam categories such as 
        phishing, promotional, or scam messages.

    \item \textbf{Limited explainability:} Users and administrators are rarely provided with 
        understandable explanations for classification decisions, reducing trust in automated 
        filtering systems.

    \item \textbf{Model rigidity:} Single-model approaches struggle to generalize across diverse 
        message structures and evolving spam patterns~\cite{dietterich2000ensemble}.
\end{enumerate}
\enumitemEndSpacing{}

Therefore, there is a need for a robust SMS filtering system that supports multi-category classification 
while providing interpretable and explainable outputs.

% Project Objectives
\section{PROJECT OBJECTIVES}

The primary objective of this project is to design and implement \textbf{\Project{}}, 
an explainable and extensible SMS classification system. 

The specific objectives are:
\begin{itemize}
    \item To collect and preprocess SMS data suitable for multi-category classification.
    \item To extract meaningful textual features using statistical and sequential representations.
    \item To train and evaluate multiple machine learning and deep learning models, including Logistic Regression, Naive Bayes, Support Vector Machines, and Recurrent Neural Networks.
    \item To design an ensemble-based result aggregation mechanism for improved prediction robustness~\cite{dietterich2000ensemble}.
    \item To integrate XAI techniques that provide human-interpretable explanations for each prediction~\cite{ribeiro2016why, lundberg2017unified}.
\end{itemize}
\enumitemEndSpacing{}

% Scope of the Project
\section{SCOPE OF THE PROJECT}

The scope of this project includes:
\begin{itemize}
    \item SMS messages written in the English language.
    \item Offline model training and evaluation using publicly available datasets.
    \item Explainability at the word or token level for classification decisions.
\end{itemize}
\enumitemEndSpacing{}

The project does not address multilingual SMS detection, real-time telecom network deployment, 
or encrypted message platforms.

% Significance of the Project
\section{SIGNIFICANCE OF THE PROJECT}

By combining ensemble learning with explainable AI techniques, this project aims to improve both 
the accuracy and transparency of SMS spam detection systems. The proposed solution benefits:
\begin{itemize}
    \item \textbf{End users}, by providing understandable reasons for message blocking.
    \item \textbf{System administrators}, by enabling debugging and model auditing.
    \item \textbf{Researchers}, by offering a modular framework for experimenting with hybrid models and XAI techniques.
\end{itemize}
\enumitemEndSpacing{}
