% -------------- RISK ANALYSIS AND MITIGATION --------------
\chapter{RISK ANALYSIS AND MITIGATION}%
\label{chap:risk_analysis_and_mitigation}

% Introuction to Risk Analysis and Mitigation
\section{INTRODUCTION}

Risk analysis is a critical component of the Software Development Life Cycle (SDLC) that 
identifies potential uncertainties which may negatively impact project objectives. Early 
identification and mitigation of risks improves project reliability and success 
rate~\cite{sommerville2011software}. This chapter discusses the major risks associated with the 
development of the proposed SMS Spam Shield system and outlines appropriate mitigation strategies.

%Risk Identification
\section{RISK IDENTIFICATION}

The risks identified in this project are categorized into technical, data-related, operational, and ethical risks.

% Technical Risks
\subsection{Technical Risks}

Technical risks pertain to challenges in model development, integration, and performance:
\begin{itemize}
    \item \textbf{Model performance degradation:} Trained models may fail to generalize to unseen SMS patterns.
    \item \textbf{Overfitting:} Complex models such as LSTM may overfit limited datasets.
    \item \textbf{System integration issues:} Difficulty in integrating multiple models and explainability tools.
\end{itemize}
\enumitemEndSpacing{}

% Data-Related Risks
\subsection{Data-Related Risks}

Data-related risks involve issues with the quality and representativeness of the SMS dataset:
\begin{itemize}
    \item \textbf{Class imbalance:} Uneven distribution of SMS categories can bias model predictions.
    \item \textbf{Data quality issues:} Noisy or mislabeled data may degrade classification accuracy.
\end{itemize}
\enumitemEndSpacing{}

% Operational Risks
\subsection{Operational Risks}

Operational risks relate to deployment and maintenance challenges:
\begin{itemize}
    \item \textbf{Computational constraints:} Limited hardware resources may restrict model training or inference speed.
    \item \textbf{Scalability limitations:} The system may not scale efficiently for high message volumes.
\end{itemize}
\enumitemEndSpacing{}

% Ethical and Explainability Risks
\subsection{Ethical and Explainability Risks}

Ethical risks arise from the need for transparency and user trust in automated spam detection:
\begin{itemize}
    \item \textbf{Lack of transparency:} Users may distrust automated decisions without clear explanations.
    \item \textbf{Privacy concerns:} SMS data may contain sensitive personal information.
\end{itemize}
\enumitemEndSpacing{}

% Risk Mitigation Strategies
\section{RISK MITIGATION STRATEGIES}

Table~\ref{table:risk_mitigation} summarizes identified risks and corresponding mitigation approaches.

\begin{table}
\centering
\begin{tabular}{|p{3cm}|p{4cm}|p{7cm}|}
\hline
\textbf{Risk Category} & \textbf{Identified Risk} & \textbf{Mitigation Strategy} \\
\hline
Technical & Model overfitting & Apply cross-validation, regularization, and early stopping techniques \\
\hline
Technical & Integration complexity & Adopt modular architecture and incremental integration \\
\hline
Data & Class imbalance & Use resampling techniques and class-weighted loss functions \\
\hline
Data & Poor data quality & Perform data cleaning and manual verification of samples \\
\hline
Operational & Hardware limitations & Optimize models and use lightweight classifiers for inference \\
\hline
Operational & Scalability issues & Design stateless inference modules and batch processing \\
\hline
Ethical & Lack of transparency & Integrate explainable AI techniques (LIME, SHAP) \cite{ribeiro2016why} \\
\hline
Ethical & Privacy concerns & Anonymize data and avoid storing personally identifiable information \\
\hline
\end{tabular}
\caption{Risk Analysis and Mitigation Strategies}%
\label{table:risk_mitigation}
\end{table}

% Residual Risk Assesment
\section{RESIDUAL RISK ASSESMENT}

Despite mitigation measures, some residual risks may remain, particularly due to evolving spam 
patterns and data drift. These risks are monitored through periodic model retraining and 
performance evaluation.

% Summary
\section{SUMMARY}

Effective risk management ensures system robustness, ethical compliance, and long-term sustainability. 
The integration of explainability mechanisms further reduces user trust-related risks and supports 
responsible deployment of machine learning systems.
